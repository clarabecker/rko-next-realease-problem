%%%%%%%%%%%%%%%%%%%%%%%%%%%%%%%%%%%%%%%%%%%%%%%%%%%%%%%%%%%%%%%%%%%%%%
% How to use writeLaTeX: 
%
% You edit the source code here on the left, and the preview on the
% right shows you the result within a few seconds.
%
% Bookmark this page and share the URL with your co-authors. They can
% edit at the same time!
%
% You can upload figures, bibliographies, custom classes and
% styles using the files menu.
%
%%%%%%%%%%%%%%%%%%%%%%%%%%%%%%%%%%%%%%%%%%%%%%%%%%%%%%%%%%%%%%%%%%%%%%

\documentclass[12pt]{article}

\usepackage{sbc-template}

\usepackage{comment}

\usepackage{graphicx,url}
\usepackage{amsmath}

\usepackage[brazil]{babel}   
%\usepackage[utf8]{inputenc}  

     
\sloppy

\title{Random Key Optimization for the Next Release Problem}

\author{Clara dos Santos Becker\inst{1}, Lucas Gabriel Falcade Nunes\inst{1}, Thassyana Snely Alves\inst{1} }

\address{Universidade do Estado de Santa Catarina 
  (UDESC)\\
  Rua Dr. Getúlio Vargas, 2822, Bela Vista -- Ibirama -- SC -- Brazil
}

\begin{document} 

\maketitle

\begin{abstract}
  This meta-paper describes the style to be used in articles and short papers
  for SBC conferences. For papers in English, you should add just an abstract
  while for the papers in Portuguese, we also ask for an abstract in
  Portuguese (``resumo''). In both cases, abstracts should not have more than
  10 lines and must be in the first page of the paper.
\end{abstract}
     
\begin{resumo} 
  Este meta-artigo descreve o estilo a ser usado na confecção de artigos e
  resumos de artigos para publicação nos anais das conferências organizadas
  pela SBC. É solicitada a escrita de resumo e abstract apenas para os artigos
  escritos em português. Artigos em inglês deverão apresentar apenas abstract.
  Nos dois casos, o autor deve tomar cuidado para que o resumo (e o abstract)
  não ultrapassem 10 linhas cada, sendo que ambos devem estar na primeira
  página do artigo.
\end{resumo}


\section{Introdução}

    A otimização é um conceito da ciência que busca explorar e encontrar soluções viáveis à tomada de decisão. As variáveis do problema devem respeitar um conjunto de restrições definidas, para então maximizar ou minimizar a função objetivo, ou seja, a motivação da aplicação.

    Um dos grandes desafios da Engenharia de Software é a identificação de prioridade e seleção das novas funcionalidades do sistema. A área de Engenharia de Requisitos (ER) identifica e documenta as necessidades dos clientes para orientar o desenvolvimento do sistema. O problema da próxima versão (NRP, de\textit{ Next Release Problem}) proposto por \cite{bagnall2001next} é um problema da otimização combinatória que visa a tomada de decisão na Engenharia de Software. Especificamente a área de ER, o levantamento de requisitos funcionais de um cenário que define as prioridades e necessidades do cliente para a definição da próxima versão do sistema. Dessa forma, duarente entrevista com o cliete, este solicita as funcionalidades desejadas, atribuindo um valor de importância chamado peso. Assim, o objetivo é selecionar um conjunto que satisfaça os clientes da melhor maneira, respeitando a restrição de custo da organização.

    O presente trabalho apresenta uma aplicação \textit{Next Release Problem} com o uso do paradigma da otimização por chaves aleatórias. Para a resolução do problema, foram aplicados algoritmos como BRKGA \cite{resende2010brkga} e GRASP \cite{chaves2024randomkey}, atráves do framework \textit{Random Key Optimizer} 
     (RKO) proposto por \cite{chaves2024rko}. Também são apresentados resultados do modelo de programação linear inteira segundo modalagem do Problema da Mochila \cite{salkin1975knapsack}.

    O trabalho estrutura-se em seções, Sendo assim, a Seção \ref{rko} apresenta a apresenta o funcionamento básico e o funcionamento da otimização por chaves aleatórias. A Seção \ref{problema} exemplifica o problema da próxima versão e apresenta a formulação matemática  A Seção \ref{resultados} explora e discute os resultados obtidos nos algoritmos. Por fim, a Seção \ref{consideracao} apresenta as considerações finais do estudo, apresenando os trabalhos futuros. 


\section{Otimização por chaves aleatórias} \label{rko}

    

\section{O problema da próxima versão} \label{problema}

   % Selecionar um conjunto ideal ideal de requisitos que satisfaça os clientes da melhor maneira e respeite a restrição de custo da organização

    O problema da próxima versão aplica-se na ES como programação linear inteira, sendo considerado um problema NP-Difícil. A proposta é um conjunto de $n$ requisitos, $m$ clientes e um um custo $c$. Sendo $c_i$ o custo que cada requisito $i \in  [n]$ e  $w_j$ o peso de cada cliente $j \in [m]$. Atribui-se ao $Q$ o conjunto de pares ($i, j$) onde o cliente $j$ solicita o requisito $i$. Além disso, atribui-se $P$ é o conjunto de pares ($i, i'$) onde cada ($i'$) é dependente de $i$. 


     \begin{align*}
        \text{maximiza} \quad & \sum_{j=1}^{m} w_j y_j \\
        \text{sujeito a} \quad & \sum_{i=1}^{n} c_i x_i \leq b, \\
        & x_i \geq x_{i'},       && \forall (i, i') \in P, \\
        & x_i \geq y_j,          && \forall (i, j) \in Q, \\
        & x_i, y_j \in \{0,1\},  && \forall i \in [n], j \in [m].
    \end{align*}

    
    A função objetivo, visa maximizar a importância do cliente $i$ para a empresa. 

    O NRP é uma variação do conhecido Problema da Mochila \textit{(Knapsack Problem}) atribuída a \cite{bagnall2001next} e \cite{salkin1975knapsack}. 


    %Clara: apresentar codificação e decodificação através de chaves aleatórias 


\section{Experimentos e resultados}\label{resultados}


\section{Considerações finais}\label{consideracao}


\bibliographystyle{sbc}
\bibliography{sbc-template}

\end{document}
